\section{Literature Review}
This section describes the current trends in RE and ML. It will define a few problems that have
been solved or automated using ML. Currently, I am working on a survey paper of RE and ML.
This literature includes only related papers to the problem statement, which I have proposed in this
document. The full literature review is under progress.
\subsection{Requirement classification}
As per our on-going survey findings that we are writing, the most highlighted problem in RE
recently is requirement analysis. It deals with the classification of functional and non-functional
requirements. Requirements documents are usually written in natural language and
contain hundreds of requirements. It is hard and time-consuming to classify them manually.
These classifications are not limited to only FR and NFR, but also to the subcategories of NFR
and quality attributes. This problem area can be categorized on the basis of existing datasets in
the literature. In our categorization, we identify the following three classes of datasets:
\begin{itemize}
\item{App Stores}
\item{Social Media}
\item{Internal Application or Software}
\end{itemize}
\subsubsection{App Stores: }
Millions of users share their reviews on app stores after downloading and using apps. They not only rate the apps but also write about the liked and disliked features. It is not easy to
find the requirements out of such a complex data. New features identification, classification of
FR and NFR, and summaries of the reviews for the improvements of an app is often done with ML
\cite{deocadez2017} \cite{lu2017automatic} \cite{maalej2015bug} \cite{jiang2014}. \\

	For the automated classification of FR and NFR, \cite{deocadez2017}  used a total of 932,338 online
reviews of the 40 top paid and free apps on appstores from top 10 different categories. Semi-supervised algorithm self- training, RASCO, Rel-RASCO for self-labeling is being used. This
semi supervision technique overcomes the manual annotation problem and shows that only
small amount of labeled data can achieve high accuracy. Naïve Bayes classification performs better than kNN, C4.5, and SMO.\\

	Another study on the classification of FR
and NFR was performed in \cite{lu2017automatic} on 6,696 raw user reviews from
iBook and 4400 raw user reviews from WhatsApp. It uses concept of
augmentation of user reviews and effectively improves user reviews classification results by
adding textual semantics to the sentences. The user reviews are augmented by several similar
words for better classification of results. The bagging algorithm outperformed Naïve Bayes and J.48.\\

	A classification method is produced in \cite{maalej2015bug} for identifying
bug reports and feature requests from user reviews. Total 146,057 reviews for 40 apps were
collected from Appstore and Google Play Store. For further experimentation, 4,400 reviews
are selected. The proposed model shows the upwards of 70 percent precision and 80 percent
recall could be obtained using multiple binary classifier, as an alternative to a single multiclass
classifier. For the classification, binary Naive Bayes algorithm is used. The results show that
the commonly used NLP techniques such as stop word removal and lemmatization could
negatively affect the performance of this classification task.\\

	In the paper \cite{jiang2014}, the primary goal is to transform online reviews into evolutionary requirements.
Karplersky internet security 2011 from Amazon and mobile app of Tune-In Radio Pro V3.6
from the appstore are taken as dataset. The characteristic analysis of the reviews is
considered for the automated task analysis and software relation-based propagation approach
(SRPA) technique is used for the identification of opinion about common software features. Each set
manually labeled the potential software feature, opinion and the polarities in the reviews, and
then classify the reviews on the basis of relevant opinion semantics. For clustering, the opinion
expression network algorithm Grivan Newman (GN) is used in the proposed methods S-GN. The GN algorithm produces optimized number of clusters. For the second problem, the system helped the developer and proposed a set of related evolutionary requirements for the system. This problem in this study makes it more interesting by suggesting the relevant and essential user requirements to a developer by adding polarization factor. That is not used
commonly in this domain.\\

\subsubsection{Social Media: }
Social media e.g. Twitter and Facebook have become popular platforms to gather the
requirements from user posts. Users are sharing their new features requests, feedback and bug
report on social platforms. Previous studies \cite{Singer:2014} \cite{Prasetyo} \cite{Achananuparp:2012} in software engineering have applied different techniques to analyzed tweets contents related to software development. These tweets were posted by developers describing the relevant and irrelevant programming languages, libraries system methodologies tweets for software development. Our study is focused on getting user feedback for software and mobile applications for obtaining user requirements. ML has been used to classify tweets into meaningful
categories like new requests and bugs etc. \cite{guzman2017} \cite{williams2017} . These classifications and the information
from these tweets helped the industry to know the user feedback for software improvements.\\

	ALERTme \cite{guzman2017} approach proposes for classifying, grouping and ranking tweets in software
evolution process. For this, a total of 68,108 tweets i.e. collection of two months of tweets about
Spotify, Dropbox, and Slack software dataset are used. The output is binary classification i.e.
improvement request or other. For the automated classification, supervised learning algorithm Multinomial Naïve Bayes is used.  
Further improvement requests are considered for the
grouping which helped to sort the request and summarize them accordingly. The major contribution helped to reduce human
effort to analyze each tweet for eliciting the requirements and knowing the issues of software. As the last step, these summaries and tweets are ranked by high
worthy tweets. A drawback of this study is the high number of manual annotations for labeling
the data as request or others. Besides, the majority voting scheme is being used to solve the
disagreements. Three annotators did this process, and it took around 13.5 hours for each
annotator to complete the task.\\

	The next study \cite{williams2017} classify and summarize the tweets. A total of 
4,000 tweets are randomly selected for ten different software. The proposed model
classifies them into bug, requirement, and spam using Naïve Bayes and SVM. The results
showed that 50\% of data contained useful technical feedback and achieved an average
classification F1 of 72\% using SVM that is better than state of the art in the literature  \cite{guzman2017}. The
reason is the feedback dedicated to technical stakeholders i.e. developer related tweets are
focused and analyzed. Tweets dataset is labeled manually. Different techniques with vector space model (VSM )and
NB are used. VSM is used for preprocessing e.g. stop word removing, sentimental analysis, stemming,
and Bag of words. However, the results show that these parameters do not
improve to help the results of ML algorithm for classification of the tweets. Unlike the political
tweets which are polarized and carry emotions, software tweets are neutral in nature.\\

\subsubsection{Internal Application or Software: }
The third category is dealing with requirement classification using the software product data
i.e. written in Software Requirements Specification (SRS). It could be dataset for some interanl used software system. This data is composed of different software functional and nonfunctional
requirements.  The nonfunctional requirements have subcategories that include availability,
fault tolerance, legal, look and feel, maintainability, operational, performance, portability,
scalability, security and usability etc. This dataset is provided by the Requirement Engineering
(RE) conference and named as Quality attributes (NFR) dataset. The size of the total data set
has 625 requirements with 225 FR and rest with NFRs subcategories. \\

	For the automated classification of FR and NFR and identification of the subcategories of NFR \cite{kurtanovic2017}, support
vector machine (SVM) algorithm is used. The data is not equally distributed because of the small
number of requirements in NFRs subcategories, which are ignored. Data
under-sampling problem is solved by using external data i.e. user comment
dataset from the Amazon, and a hybrid approach is proposed with the new dataset. This
dataset contains performance and usability requirements. For the identification of specific
NFRs, proposed methodology achieve the highest precision and recall for security and
performance NFRs with ~92\% precision and ~90\% recall. \\

	Two main goals are targeted in \cite{Abad},
first is classifying the FRs and NFRs, and second is an identification of NFR category. The data is
pre-processed as a first step. Feature co-occurrence and regular expression are used to increase
the weight of influential words used in NFR. The supervised learning algorithm J.48 DT is used
for classifying FRs and NFRs. For achieving the categorization or classification of NFR,
topic modeling using unsupervised algorithm LDA and BTM is applied. For the topic generation, the results show  that BNB works better than clustering, k-means, LDA, BTM. \\


	Another study for solving the same problem with NFR dataset with additional security dataset is conducted \cite{Dekhtyar}.
Software requirements are classified with the focus on security related requirements. ML algorithm
conventional neural network (CNN) with a specific setting in Tensor-Flow helped to achieve the goal and better results. In all
these classification problems, human input is involved for the annotation of the requirements.
Semi-supervision requires less human effort in labeling requirements than fully supervised
methods. The semi-supervised approach \cite{Casamayor} using Naïve Bayes resulted in accuracy rates
above 70\%, considerably higher than the results obtained with supervised methods using
standard collections of documents.
\begin{table}[]
\begin{tabular}{|p{1.2cm}|p{4.5cm}|p{2cm}|p{2cm}|p{2cm}|p{2cm}|p{2cm}|}
\hline
\textbf{Paper No.}     & \textbf{Problem}                                                 & \textbf{Algorithm}                                                           & \textbf{Dataset}        & \textbf{Mannual Annotation} & \textbf{Hybrid datasources} & \textbf{Evaluation} \\ \hline
\cite{deocadez2017}    & classication of FR and NFR                                       & Naive Bayes                                                                  & appstore                & ($\circ$)                     & x                              & x                                   \\ \hline
\cite{lu2017automatic} & classication of FR and NFR                                       & Bagging                                                                      & appstore                & \checkmark                  & x                              & x                                   \\ \hline
\cite{maalej2015bug}   & classification as bug reports and feature requests               & Naive Bayes                                                                  & appstore & \checkmark                  & x                              & x                                   \\ \hline
\cite{jiang2014}       & Identify evolutionary requirements                               & \begin{tabular}[c]{@{}c@{}}Clustering\\  with\\ network algo\end{tabular}    & Amazon and appstore     & \checkmark                  & \checkmark                     & x                                   \\ \hline
\cite{guzman2017}      & classifying, grouping and ranking as improvement reques or other & Naive Bayes                                                                  & Tweets                  & \checkmark                  & x                              & x                                   \\ \hline
\cite{williams2017}    & classifying and summarize as bug, requirement, and spam          & Naive Bayes                                                                  & Tweets                  & \checkmark                  & x                              & x                                   \\ \hline
\cite{kurtanovic2017}  & classification as FR and NFR and NFRs subcategories              & SVM                                                                          & PROMISE                 & \checkmark                  & x                              & x                                   \\ \hline
\cite{Abad}            & classification as FR and NFR and NFRs subcategories              & J.48, Naive Bayes                                                         & PROMISE                 & \checkmark                  & x                              & x                                   \\ \hline
\cite{Dekhtyar}        & classification as FR and NFR and security related req.           &  CNN
 & PROMISE                 & \checkmark                  & x                              & x                                   \\ \hline
\cite{Casamayor}       & classification as FR and NFR and NFRs subcategories              & Naive Bayes                                                                  & PROMISE                 & ($\circ$)                           & x                              & x                                   \\ \hline
\cite{Nayebi}          & classification as bug reports, feature request, and other        & Naive Bayes, SVM.                                                         & Twitter and appstore    & \checkmark                  & \checkmark                     & x                                   \\ \hline
\end{tabular}
\caption{Table1. Summary of state of the art}
\label{table:LiteratureReview}
\center{Legend:  (\checkmark)   present in the study; (x) not present in the study; ($\circ$) partially present}
\end{table}
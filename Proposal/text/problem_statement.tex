\section{Problem Statement}
Conventionally, software requirement elicitation and analysis are only limited to meetings,
interviews, and documented data etc. All the tasks are performed manually, requiring more
effort and time. With recent data trends from various sources, the user satisfaction and opinion
are more integrated into the industry. Recent studies mentioned in literature review section have shown that the user analytics tools
and techniques are helping developers and practitioners to deal with a large number of user
feedback by filtering, classifying, and summarizing them, to decide what requirements and
features they should add, change or eliminate. Different ML algorithms have been used for
making it automated with different settings and parameters. Based on the literaure gap mentioned in the section above my reserach questions are:\\


\textbf{RQ1: Can advanced classifiers or advance settings perform better than the ones used in the existing
literature?}\\
Out of various classifiers, it has been found in the literature that SVM and NB perform well for
short texts. Since the user reviews about software systems are short texts, these two algorithms
are usually preferred. However, it could be researched that how advanced classifiers or settings perform
in comparison to SVM and NB. The comparison could be drawn on the basis of the ability of a
classifier to process larger datasets, the processing time, complexity analysis and accuracy of
classification. I am planning to apply neural network algorithm. However, it majorly depends on my dataset.\\

\textbf{RQ2: What would be the impact of using semi-supervised in
contrast to supervised learning?}\\
One of the problems with the classifiers usually used SVM and NB is that they fall into the
category of supervised learning. Supervised learning classifiers require a lot of human input as
they work on the basis of provided training data. The training data needs to be classified by
humans which is time consuming, particularly for large datasets such as the ones we are
discussing. In such scenarios, semi-supervised and unsupervised learning would require
less human input. Such techniques might decrease the efficiency, but for larger datasets, semi-supervised
might be less time consuming than supervised learning
techniques.\\

\textbf{RQ:3 How social network sites (Twitter, Facebook and LinkedIn etc.) and online discussion
forums (app stores and Amazon) reviews complement each other for developing new
app or software?}\\
The importance of the end user involvement in today’s software is highlighted
in the literature. However, the reusability of this large dataset is not discussed and missing. The
addition of the social media data in the software development process in early stages can add
global and heterogeneous perspectives. As different people from diverse cultures all around the world are sharing reviews about app or software on social media. The benefit of the large dataset is to gather information
about the specific category and see what requirements users wished to have in both functional and
nonfunctional perspective. A developer or practitioner without enough requirements for the project
 can take benefit of the existing data knowledge for
 specific app and domain. It will give an additional set of requirements and features and
domain-specific knowledge for building the better system. I want to implement this idea to a case
study to know the impact and enhancement of requirement that can be gained by this additional
data.

	Also, some similar features mapping can be challenging. For example, in \cite{Martens:2017} performed an initial study to observe the impact of emotional sentiment on app reviews as an informative feature and its pitfall. After researching and analyzing hybrid sources, some other informative features from one source to another can be improvized.\\

\textbf{RQ4: How useful are the social network sites (Twitter, Facebook, LinkedIn) and online
discussion forums (app stores, Amazon, google stores) reviews for requirement elicitation
and what works better?}\\
Users discuss software systems on numerous platforms including social media websites,
app stores, and online discussion forums etc. However, all the discussions do not contain
information specific to requirement engineering. Any information that doesn’t fall into the
category of features requests or bug reports generally is not of interest to us. Therefore, it is
necessary to evaluate the percentage of relevant information in each of these platforms. A
comparative study could be useful to evaluate which of these platforms contain more useful
information as compared to rest of the platforms. Also, the hybrid models containing different
sources e.g. Twitter and Amazon reviews, Twitter and app stores. What combination works better
and provide more features or bug report.\\

	The datasets obtained from online platforms are huge. The content shared by the users in large number. However, the information would be more or less relevant to similar
functionalities of the application. Going through all the extracted dataset that is classified as
feature report or bug report would be humanly impossible. It is therefore imperative to
summarize the obtained information. The summarization should not only group similar requests
or bug reports, but also emphasize on the most discussed issues. For example, if there is a bug
with a camera filter, and is discussed various times, it should be emphasized in the same manner
to the developer, indicating that it requires more or immediate attention.\\

	In this area after classification and all experiments no study has shown what was the impact 
of their finding and how much results were useful for devolper or practioneers. I want to perform some study with help of  case study that can help to identify the impact
 of these finding in industry. The other missing feature in the literature is adaptability of the tool in the industry. For making
this concept more practical and useful in industry there should be some tool or plugin with
the entire features. The ongoing survey disclosed the fact that tools are not available or if available only
taking data from one source. For this gap, plugin can be devolped. That plugin can be added
to the existing RE tools used in the industry.
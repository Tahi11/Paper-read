%\subsubsection {Evaluation\\}
One of the identified gaps in my research is the lack of evaluation for the existing studies. Most studies did not map their experiment results on the industry needs and impacts. This means the obtained results are similar to the manual study performed in the industry, if yes then how much? Also, how efficient results automated systems are providing to the users by considering the extra information and non-relevant information? I plan to perform this evaluation study as given below: \\
\begin {itemize}
\item{The first method to evaluate the results is to draw a comparison between two different versions of one application. The specific application data will be collected before the new version. All the requirements given by the proposed system will be matched in the new version. Each identified requirement exist in the new version or not will be checked manually.\\}
\item{The second method is the automated version of the first method. For this two dataset before and after the new version will be compared. This method will compare which requirements are mentioned by users in the first dataset i.e. before the new version. A traceability link for these requirements will be generated in the second dataset i.e. after the new version. If users have stopped asking about these requirements after the release, we will assume that those requirements are now part of the application in the new version.\\}
\item{In the third method, I will perform an industrial case study that will investigate the difference in the developer requirements list result and my tool result list. The developer will mark the similar requirements in both results and give comments on comparisons. From these results, we can see the correctness and effectiveness of our system. This study will help to provide a better insight into the comparison due to the addition of comments. Which will ultimately help to improve the evaluation of our study.\\}
\end{itemize}
Our two initial evaluation case studies have some threats to validity such as if requirement identified by my proposed system is not developed in the new version due to some constraints e.g. organizational policies, finance, and scope of application etc. The third method will add the comments of developers in the results for better analysis and more insight.